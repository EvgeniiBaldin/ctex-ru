%  -*- coding: koi8-r -*-
\documentclass[a4,12pt]{article} 
\usepackage[warn]{mathtext}
\usepackage[T2A]{fontenc}
\usepackage[koi8-r]{inputenc}
\usepackage[english,russian]{babel}
\usepackage{indentfirst}%first paragraph indent
\usepackage{graphicx}
\usepackage{amssymb}
\usepackage{amsmath}
\usepackage{wrapfig}
\usepackage{afterpage}
\usepackage{srcltx}
\usepackage{latexsym} 
%\usepackage{sparticles}         %Package for displaying sparticle names. 
\usepackage{feynmp}             %Package for feynman diagrams. 
% : -> . in caption 
\usepackage[centerlast,small]{caption2}
\renewcommand{\captionlabeldelim}{.}
\usepackage{xspace}

%backslash
\newcommand{\bs}{\symbol{'134}}
%degree
\newcommand{\grad}{\ensuremath{{}^{\circ}}\xspace}

\newcommand{\ee}{\ensuremath{{e^{+}e^{-}}}\xspace}
\newcommand{\mumu}{\ensuremath{{\mu^{+}\mu^{-}}}\xspace}
\newcommand{\tautau}{\ensuremath{\tau^{+}\tau^{-}}\xspace}
\newcommand{\pipi}{\ensuremath{\pi^{+}\pi^{-}}\xspace}
\newcommand{\lele}{\ensuremath{\ell^{+}\ell^{-}}\xspace}
\newcommand{\hadr}{\ensuremath{hadrons}\xspace}
\newcommand{\JP}{\ensuremath{J/\psi\,}\xspace}
\newcommand{\PP}{\ensuremath{\psi'}\xspace}
\renewcommand{\epsilon}{\varepsilon}


\begin{document}
\thispagestyle{empty}
\begin{fmffile}{ee-psi-ee}
  \fmfframe(1,7)(1,7){ 
   \begin{fmfgraph*}(110,62) %Sets size of Diagram
    \fmfleft{ei,pi}     %Sets there to be 2 sources 
    \fmfright{eo,po}    %Sets there to be 2  outputs
    \fmflabel{$e^-$}{ei} %Labels one of the left sources
    \fmflabel{$e^+$}{pi} %Labels one of the left sources
    \fmflabel{$e^+$}{po} %Labels one of the right outputs
    \fmflabel{$e^-$}{eo} %Labels one of the right outputs
    \fmf{fermion}{ei,Ji,pi} %Connects the sources with a vertex.
    \fmf{fermion}{po,Jo,eo} %Connects the outputs with a vertex.
    \fmflabel{$\Gamma_{\ee}$}{Ji} %Labels 
    \fmflabel{$\text{Br}_{\ee}$}{Jo} %Labels 
    \fmf{heavy,label=$J/\psi$}{Ji,Jo} %Labels the conneting line.
   \end{fmfgraph*}
  }
\end{fmffile}
\end{document}

